%"Modelado de la Propagación de Malware: Un Enfoque SEIARS para la Dinámica de Dispositivos Asintomáticos"


\section{Formulación del Modelo}
El modelado matemático es una de las herramientas más importantes utilizadas para comprender la dinámica de la transmisión de virus. El modelo desarrollado considera el caso en el que los usuarios de dispositivos que han sido infectados con software malicioso pueden ser controlados sin que ellos lo sepan. El objetivo de tal ataque es controlar el sistema del dispositivo de manera anónima y aumentar el alcance de los ataques. En la jerga epidemiológica, estos tipos de dispositivos se llaman dispositivos asintomáticos. Análogamente a la propagación de alguna enfermedad en una población, los dispositivos asintomáticos pueden portar la infección sin mostrar síntomas. Son infecciosos y contribuyen a la distribución del virus. Sin embargo, son mucho más difíciles de detectar ya que no muestran síntomas.

En este sentido, se propone un modelo compartimental SEIARS donde S representa Susceptibles, E Expuestos, I Infecciosos, A Asintomáticos y R Recuperados, como se muestra en la Figura \ref{fig:SEIARS_model}. La notación para el modelo SEIARS se resume en la Tabla \ref{tab:SEIARS_notation}. Se incluye un compartimento asintomático además del compartimento infeccioso, y se denota por A. Se asume que son infecciosos a una tasa de transmisión reducida $\beta_a$, como se representa mediante una línea discontinua en la Figura \ref{fig:SEIARS_model}. Los dispositivos Expuestos están infectados pero aún no son infecciosos y pueden adquirir inmunidad temporal al ataque de malware a una tasa de $\sigma$. Si el software de seguridad puede detectar y eliminar el malware, entonces los dispositivos Asintomáticos e Infecciosos adquirirán inmunidad temporal a tasas $\delta_a$ y $\delta_i$, respectivamente. De lo contrario, los dispositivos serán eliminados del sistema a tasas $\mu$ y $\mu_a$, donde $\mu$ denota la tasa normal de daño del dispositivo, y $\mu_a$ y $\mu_i$ son las tasas de daño del dispositivo debido a un ataque de los compartimentos A e I, respectivamente. Finalmente, los dispositivos recuperados pierden su inmunidad temporal y regresan al compartimento susceptible a una tasa de $\gamma$. La dinámica del modelo en la Figura \ref{fig:SEIARS_model} está gobernada por el conjunto de ecuaciones diferenciales ordinarias dadas en la ecuación 1.

\begin{figure}[H]
    \centering
    \includegraphics[width=0.3\textwidth]{include/logo.png}
    \caption{Diagrama de flujo del modelo SEIARS}
    \label{fig:SEIARS_model} % Cambié 'tab' a 'fig'
\end{figure}

\[
\begin{cases}
    \cfrac{dS}{dt} =\Lambda - \beta S(I+\lambda A) - \mu S + \eta R, \quad&S(0)=S_0\\
    \cfrac{dE}{dt} =\beta S(I+\lambda A) - (\alpha + \mu)E, \quad&E(0)=E_0 \\
    \cfrac{dI}{dt} =\rho_1\alpha E - (\psi+\mu+\xi_2)I, \quad&I(0)=I_0 \\
    \cfrac{dA}{dt} =(1-\rho_1-\rho_2)\alpha E - (\delta+\mu+\xi_1)A, \quad&A(0)=A_0 \\
    \cfrac{dR}{dt} =\psi I + \delta A+\rho_2\alpha E - (\eta+\mu)R, \quad&R(0)=R_0 \\
\end{cases}
\]

\begin{table}[ht]
    \centering % Añadido para centrar la tabla
    \begin{tabular}{c| l}
        Símbolo & Explicación \\ \hline
        $S(t)$ &  Número total de computadoras Susceptibles en el tiempo $t$.\\
        $E(t)$ &  Número total de computadoras Expuestas en el tiempo $t$.\\
        $I(t)$ &  Número total de computadoras Infectadas en el tiempo $t$.\\
        $A(t)$ &  Número total de computadoras Asintomáticas en el tiempo $t$.\\
        $R(t)$ &  Número total de computadoras Recuperadas en el tiempo $t$.\\
        $\Lambda$ & Número de nuevos dispositivos que entran en el sistema.\\
            & Se llama tasa de natalidad. \\ 
        $\mu$   &  Tasa a la cual un dispositivo falla debido al número de veces que ha sido utilizado.\\
                & Se llama tasa de muerte natural.\\ 
        $\lambda$ & Tasa de infección de un dispositivo Asintomático. \\ 
        $\xi$ &  Tasa a la cual un dispositivo se daña debido a un ataque que ocurre en la etapa\\
              & Asintomática $(\xi_1)$ o Infectada $(\xi_2)$.\\ 
        $\beta$ &  Tasa de contacto\\ 
        $\rho_1\text{ y }\rho_2$ & Proporciones a las que los dispositivos dejan el compartimento Expuesto \\
                & para pasar a la etapa Infectada o Recuperada, respectivamente \\ 
        $\alpha$ &  Tasa a la cual los dispositivos dejan el compartimento Expuesto, ya sea hacia la etapa \\
            & Infectada, Recuperada o Asintomática con las proporciones $\rho_1,\rho_2$ y $1-(\rho_1+\rho_2)$ \\ 
            & respectivamente\\ 
        $\delta\text{ y }\psi$ & Tasas a las que los dispositivos dejan, respectivamente, los compartimentos\\
        & Asintomático e Infectado para el compartimento Recuperado. \\ 
        $\eta$ &  Tasa de recuperación a la que los dispositivos pierden su inmunidad temporal \\
            & y regresan al compartimento Susceptible.\\ 
    \end{tabular}
    \caption{Notación del modelo SEIARS}
    \label{tab:SEIARS_notation}
\end{table}