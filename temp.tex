
\section{Navegación de Directorios}

\subsection*{Sistema Operativo}

\subsubsection*{Linux}
\begin{itemize}
    \item \texttt{cd /ruta/al/directorio}: Cambiar al directorio especificado.
    \item \texttt{ls}: Listar archivos y directorios.
    \item \texttt{ls -l}: Listar con detalles.
    \item \texttt{ls -a}: Listar todos los archivos (incluyendo ocultos).
\end{itemize}

\subsubsection*{Windows}
\begin{itemize}
    \item \texttt{cd \textbackslash ruta\textbackslash al\textbackslash directorio}: Cambiar al directorio especificado.
    \item \texttt{dir}: Listar archivos y directorios.
\end{itemize}

\section*{Visualización}

\subsubsection*{Linux}
Para visualizar en forma de árbol los directorios, instalaremos un pequeño paquete llamado \texttt{tree}:

\begin{verbatim}
sudo apt install tree
tree /ruta/del/directorio
\end{verbatim}

A continuación, un ejemplo de la estructura de directorios:

\dirtree{%
.1 /home/sterben/Documentos/Python/.
.2 Finanzas/.
.3 app/.
.4 app.py.
.4 board.py.
.4 temp/.
.3 data.
.3 data-journal.
.3 dat.sqbpro.
.3 INFO.txt.
.3 main.py.
.3 prueba.py.
.3 temp/.
.4 hola.txt.
.4 temp2/.
.2 MP3/.
.3 list.txt.
.3 mp3.py.
.3 music/.
.2 qr/.
.3 IUqr.py.
}

\section*{Gestión de Archivos y Directorios}

\subsubsection*{Linux}
\begin{itemize}
    \item \texttt{cp archivo destino}: Copiar un archivo.
    \item \texttt{mv archivo destino}: Mover o renombrar un archivo.
    \item \texttt{rm archivo}: Eliminar un archivo.
    \item \texttt{mkdir nombre\_directorio}: Crear un nuevo directorio.
    \item \texttt{rmdir nombre\_directorio}: Eliminar un directorio vacío.
\end{itemize}

\subsubsection*{Windows}
\begin{itemize}
    \item \texttt{copy archivo destino}: Copiar un archivo.
    \item \texttt{move archivo destino}: Mover o renombrar un archivo.
    \item \texttt{del archivo}: Eliminar un archivo.
    \item \texttt{mkdir nombre\_directorio}: Crear un nuevo directorio.
    \item \texttt{rmdir nombre\_directorio}: Eliminar un directorio vacío.
\end{itemize}

\section*{Visualización de Contenido}

\subsubsection*{Linux}
\begin{itemize}
    \item \texttt{cat archivo}: Mostrar el contenido de un archivo.
    \item \texttt{less archivo}: Ver el contenido de un archivo con paginación.
    \item \texttt{head archivo}: Mostrar las primeras líneas de un archivo.
    \item \texttt{tail archivo}: Mostrar las últimas líneas de un archivo.
\end{itemize}

\subsubsection*{Windows}
\begin{itemize}
    \item \texttt{type archivo}: Mostrar el contenido de un archivo.
    \item \texttt{more archivo}: Ver el contenido de un archivo con paginación.
\end{itemize}

\section*{Información del Sistema}

\subsubsection*{Linux}
\begin{itemize}
    \item \texttt{top}: Mostrar procesos en tiempo real.
    \item \texttt{df -h}: Mostrar espacio en disco.
    \item \texttt{free -h}: Mostrar memoria utilizada.
\end{itemize}

\subsubsection*{Windows}
\begin{itemize}
    \item \texttt{tasklist}: Mostrar lista de procesos en ejecución.
    \item \texttt{systeminfo}: Mostrar información del sistema.
    \item \texttt{wmic logicaldisk get size,freespace,caption}: Mostrar información del disco.
\end{itemize}

\section*{Red}

\subsubsection*{Linux}
\begin{itemize}
    \item \texttt{ping direccion}: Comprobar la conectividad a una dirección.
    \item \texttt{ifconfig}: Mostrar configuración de red (en sistemas modernos, usa \texttt{ip addr}).
    \item \texttt{curl url}: Realizar una solicitud HTTP.
\end{itemize}

\subsubsection*{Windows}
\begin{itemize}
    \item \texttt{ping direccion}: Comprobar la conectividad a una dirección.
    \item \texttt{ipconfig}: Mostrar configuración de red.
    \item \texttt{curl url}: (disponible en Windows 10 y versiones posteriores).
\end{itemize}

\section*{Ayuda}

\subsubsection*{Linux}
\begin{itemize}
    \item \texttt{man comando}: Mostrar el manual de un comando.
    \item \texttt{comando --help}: Mostrar ayuda rápida para un comando.
\end{itemize}

\subsubsection*{Windows}
\begin{itemize}
    \item \texttt{comando /?}: Mostrar ayuda rápida para un comando.
\end{itemize}